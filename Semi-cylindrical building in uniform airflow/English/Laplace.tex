\subsection{The big picture}

Equation~\eqref{eq55}, which we have obtained, is known as a \emph{Laplace equation}. Together with the boundary conditions of the problem, it provides sufficient information to describe a physical phenomenon associated with a time-independent potential field (i.e.\ in the steady state), such as the electrostatic field or the velocity field of a fluid.

This type of modeling appears very broadly in physics; we list a few representative examples below:
\renewcommand{\labelitemii}{*}

\begin{itemize}
    \item Particle diffusion obeying Fick's law:
    \begin{itemize}
        \item Scalar field: particle density $n$;
        \item Vector field: $\mathbf{j}_D = -D\nabla n$, where $D$ is the diffusion coefficient;
    \end{itemize}
    \item Heat diffusion obeying Fourier's law:
    \begin{itemize}
        \item Scalar field: temperature $T$;
        \item Vector field: $\mathbf{j}_k = -k\nabla T$, where $k$ is the thermal conductivity;
    \end{itemize}
    \item Electrical conduction obeying Ohm's law:
    \begin{itemize}
        \item Scalar field: electric potential $V$;
        \item Vector field: $\mathbf{j} = \sigma\nabla V$, where $\sigma$ is the electrical conductivity;
    \end{itemize}
\end{itemize}

In general, one considers a scalar function $\zeta$ satisfying the Laplace equation
\[
\nabla^2 \zeta = 0,
\]
and a vector field $\mathbf{J}$, called the flux, related to $\zeta$ by
\[
\mathbf{J} = \Upsilon \nabla \zeta,
\]
where $\Upsilon$ is a constant characteristic of the physical problem.

These properties, together with the geometric boundary conditions and imposed limits specific to each problem, imply the uniqueness of the solution to the Laplace equation $\nabla^2 \zeta = 0$ (by the uniqueness theorem).

Consequently, two problems associated with identical or analogous physical quantities, in the sense described above, and sharing the same configuration, will possess identical or analogous solutions, uniquely determined.

\subsection{General solution in cylindrical coordinates}

We now rederive the general solution of equation~\eqref{eq55} in cylindrical coordinates, which is simultaneously the general solution of any Laplace equation possessing cylindrical symmetry.

\begin{tcolorbox}[colback=blue!10, colframe=blue!50!black, title= ]
We first seek a particular solution of the form
\[
\phi = S(r)\Theta(\theta).
\]
Substituting into the Laplace operator yields
\begin{align}
    \nabla^2 \phi
    &= \frac{1}{r}\frac{\partial}{\partial r}\left(r\frac{\partial\phi}{\partial r}\right)
    + \frac{1}{r^2}\frac{\partial^2 \phi}{\partial\theta^2}=0,\\
    \implies\;
    &\frac{r}{S}\frac{d}{dr}\left(r\frac{dS}{dr}\right)
    + \frac{1}{\Theta}\frac{d^2\Theta}{d\theta^2} = 0.
\end{align}

Since $S$ depends only on $r$ and $\Theta$ depends only on $\theta$, each term must be equal to a constant:
\begin{align}
    \frac{r}{S}\frac{d}{dr}\left(r\frac{dS}{dr}\right) &= C_1,\\
    \frac{1}{\Theta}\frac{d^2\Theta}{d\theta^2} &= C_2, \label{eq4.4}\\
    C_1 + C_2 &= 0.
\end{align}

We observe that $C_2$ must be negative, since $\Theta(\theta)$ must be a periodic function. If $C_2$ were positive, $\Theta$ would be exponential in $\theta$, which is unphysical. We therefore set $C_2 = -k^2$, obtaining
\begin{align}
    \Theta(\theta) = A\cos k\theta + B\sin k\theta.
\end{align}

Furthermore, the condition $\Theta(\theta+2\pi)=\Theta(\theta)$ requires $k$ to be an integer:
\[
k = 0,1,2,3,\dots
\]

The radial equation
\[
\frac{r}{S}\frac{d}{dr}\left(r\frac{dS}{dr}\right)=k^2
\]
admits power-law solutions $S=r^n$, where $n$ satisfies
\[
r\frac{d}{dr}(rn r^{n-1}) = n^2 r^n = k^2 S,
\quad\Rightarrow\quad n=\pm k.
\]

Thus, the general solution for $k\neq 0$ is
\begin{align}
    S(r)= C r^k + D r^{-k}.
\end{align}

The case $k=0$ must be treated separately. Substituting $k=0$ gives
\begin{align}
    \frac{r}{S}\frac{d}{dr}\left(r\frac{dS}{dr}\right)=0
    \;\Rightarrow\;
    r\frac{dS}{dr}=C
    \;\Rightarrow\;
    S=C\ln r + D.
\end{align}

Substituting $k=0$ into~\eqref{eq4.4} yields
\[
\frac{d^2\Theta}{d\theta^2}=0
\;\Rightarrow\;
\Theta=A\theta+B,
\]
which is non-periodic and therefore unphysical.

The general solution is obtained as a linear superposition of all admissible solutions:
\begin{align}
    \phi(r,\theta)
    = a_0 + b_0 \ln r
    + \sum_{k=1}^{\infty}
    \Big[
    r^k\big(a_k\cos k\theta + b_k\sin k\theta\big)
    + r^{-k}\big(c_k\cos k\theta + d_k\sin k\theta\big)
    \Big].
    \label{eq57}
\end{align}
\end{tcolorbox}

\subsection{A revision}

Imposing $\phi=0$ at $r=0$, the boundary conditions~\ref{eq3} may be rewritten as
\begin{align}
    \left\{
    \begin{array}{ll}
        \displaystyle\lim_{r\to\infty}\phi = v_0 r\cos\theta, \\[12pt]
        \displaystyle\frac{d\phi}{dr}\bigg|_{r=R} = 0.
    \end{array}
    \right.
\end{align}

The first condition can only be satisfied if
\[
a_0=b_0=b_k=d_k=0,
\]
and since $\theta$ is arbitrary, one must have $k=1$.

Thus,
\[
\phi = v_0\cos\theta\left(r+\frac{d_1}{r}\right).
\]

Applying the second boundary condition yields
\[
d_1 = v_0 R^2.
\]

Hence, the velocity potential is
\begin{align}
    \phi(r,\theta)=v_0\cos\theta\left(r+\frac{R^2}{r}\right).
\end{align}

The velocity components follow as
\begin{align}
    v_r &= \frac{d\phi}{dr}
    = v_0\cos\theta\left(1-\frac{R^2}{r^2}\right),\\
    v_\theta &= \frac{1}{r}\frac{d\phi}{d\theta}
    = -v_0\sin\theta\left(1+\frac{R^2}{r^2}\right).
\end{align}

Therefore,
\begin{empheq}[box=\fbox]{align*}
    \mathbf{v}
    &= v_0\left(1-\frac{R^2}{r^2}\right)\cos\theta~\hat{\mathbf{r}}
    - v_0\left(1+\frac{R^2}{r^2}\right)\sin\theta~\hat{\boldsymbol{\theta}}.
\end{empheq}
