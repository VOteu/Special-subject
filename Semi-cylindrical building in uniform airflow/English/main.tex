\documentclass{article}
\usepackage{fontenc}
%\usepackage{avant}
\usepackage{amssymb}
\usepackage{listings}%code python
\usepackage{graphicx}
\usepackage{mathtools}
\usepackage{amsthm}
\usepackage{amsfonts}
\usepackage{tikz}
\usepackage{tcolorbox}
\usepackage{xcolor}
\usepackage{titlesec}
\usepackage{mdframed}
\usepackage{empheq}
\usepackage{color}
\usetikzlibrary{arrows.meta}
\usetikzlibrary{shapes}
\usepackage{array}
\usepackage[top=2cm,bottom=2cm,left=2.5cm,right=2.5cm,marginparwidth=1.75cm]{geometry}
\usetikzlibrary{calc, angles, quotes, intersections, positioning, patterns}
\usepackage{longfbox}
\usepackage{pgfplots}
\pgfplotsset{compat=1.18}
\renewcommand{\thefootnote}{\color{blue}[\arabic{footnote}]}
\usepackage{enumitem}
\setlength{\parindent}{0cm}
\usepackage{ mathrsfs}
\usepackage{longtable}

\usepackage{chngcntr}
\usepackage{caption}
% \captionsetup[figure]{labelsep=space,labelformat=arabicperiod,
%	font=small,
%	labelfont={color=black, bf},
%	textfont={color=black},name={}
% }

% Thiết lập màu cho các liên kết và tham chiếu
\usepackage{hyperref}
\hypersetup{
	colorlinks=true,
	linkcolor=blue, % Màu cho tham chiếu nội bộ
	urlcolor=blue, % Màu cho liên kết ngoại vi (URL)
	citecolor=blue % Màu cho các tham chiếu trong văn bản
}
\numberwithin{equation}{section}
\renewcommand{\figurename}{Figure}
\usepackage{fancyhdr}
\usepackage{float}
\usepackage{fontawesome5}
\usepackage{subcaption}
\usepackage{footmisc}
\usepackage{biblatex}


\addbibresource{myrefs.bib}

\title{Semi-cylindrical obstacle in uniform airflow}
\author{Vo Anh Tue}
\date{January 2025}


\begin{document}

\maketitle
\begin{abstract}
When studying the velocity field of an ideal incompressible fluid around an obstacle, we pay special attention to symmetric configurations (sphere, cylinder, etc.).
Although finding direct elementary solutions is not simple, by analogy with magnetostatics and electrostatics, we can obtain equivalent formal solutions. 
Through this analogy, the fluid problem can be fully analyzed mathematically, and the method can be extended to a large class of physical phenomena described by Laplace's equation.
\end{abstract}
\begin{figure}[H]
	\centering
	\includegraphics[width=0.6\textwidth]{Image/streamline.jpg}
	\caption{A cylindrical obstacle in uniform airflow}
\end{figure}
\newpage
\tableofcontents
\newpage
\section{Introduction}
\begin{figure}[H]
    \centering
    \includegraphics[width=0.8\linewidth]{Image/problem.png}
    \caption{Schematic of a long semi-cylindrical obstacle in uniform airflow}
    \label{F2}
\end{figure}
\begin{tcolorbox}[colback=blue!10, colframe=blue!50!black, title= Problem Statement]
Consider a building in the shape of a semi-cylinder of length $L$ and radius $R$, resting on an infinite flat ground, such that the length is very large compared with the radius (See \ref{F2}). From infinity, a stream of wind blows toward it with velocity $\mathbf{v}_0$. Determine the wind velocity at every point in space once the flow has reached a steady state.
\end{tcolorbox}
The crucial assumptions are that the airflow is incompressible and irrotational (with a very small vicosity coefficient). This means that the velocity field $\mathbf{v}$ satisfies the following equations, in region of $y\geq 0$: \begin{align}
    &\nabla\cdot\mathbf v = \mathbf 0, \label{eq1}\\
    &\nabla\times\mathbf v=\mathbf 0. \label{eq2}
\end{align} And the boundary conditions are: \begin{align}
    &\left\{
    \begin{array}{ll}
        \displaystyle \lim_{r \to \infty} \mathbf{v}(r, \theta) = \mathbf{v}_0, \\[12pt] 
        \mathbf{v}(R, \theta)\cdot\hat{r} = \mathbf 0.
    \end{array}
    \right.\label{eq3}
\end{align}
In the general case, equation~\eqref{eq2} takes the form
\[
\nabla \times \mathbf{v} = 2 \mathbf{\Omega},
\]
and one readily observes the analogy between the two Maxwell equations\footnote{
\begin{align*}
\nabla \cdot \mathbf{B} &=\mathbf 0,\\
\nabla \times \mathbf{B} &= \mu_0 \mathbf{j}.
\end{align*}}
in magnetostatics and the two partial differential equations describing an ideal fluid flow.

Therefore, we shall first examine the analogous magnetostatic problem using a simplified model presented in Section~\ref{II}. We then return to the fluid-mechanics via the corresponding concepts developed in electrostatics in Section~\ref{III}. Finally, we conclude by solving the Laplace equations to verify the accuracy of the simplified models employed, while at the same time providing the most general estimation possible.
\label{I}
\newpage
\section{Long superconducting cylinder}\label{II}
\subsection{The equivalent model}
A perfect conductor is a conductor with an infinite conductivity. This requires that the electric field inside its body is zero. A direct consequence of this property is that the magnetic field inside the object needs to be constant. 
A superconducting material is a perfect conductor with zero magnetic permeability, i.e. $\mu = 0$. 
This means that the magnetic field inside a superconductor is always zero.
 Hence, every current distribution only can reside outside or on the surface of a superconductor, 
 and the perpendicular component of the magnetic field on the object's surface is eliminated (which is called Meissner's effect).
\begin{figure}[H]
    \centering
    \includegraphics[width=0.5\textwidth]{Image/4fieldphysic.jpg}
    \caption{Meissner's effect}
    \label{F3}
\end{figure}
Considering a superconducting cylinder of infinite length, radius $R$ in the presence of a uniform magnetic field $\mathbf{B}_0$ directed perpendicularly to its axis.
When the steady state is reached, on the surface of the cyliner emerges bound surface currents generating a supplementary magnetic field equivalent to $-\mathbf{B}_0$, so that the net magnetic field is zero, inside the object and changing the field outside.
Differential equations and boundary conditions describing the system in this state are:
\begin{align}
    &\left\{
    \begin{array}{ll}
        \nabla\cdot\mathbf B=\mathbf 0, \\[12pt] 
        \nabla\times\mathbf{B} = \mathbf 0,
    \end{array}
    \right.
    &\left\{
    \begin{array}{ll}
        \displaystyle \lim_{r \to \infty} \mathbf {B}(r, \theta) = \mathbf{B}_0, \\[12pt] 
        \mathbf{B}(R, \theta)\cdot\hat{r} = \mathbf 0.
    \end{array}
    \right.\label{eq4}
\end{align} 
Obviously, the solution of this problem must be similar to that of the previous problem in their form cause their governing equations and boundary conditions are identical in forms. \newline
To continue, we will offer some guesses and assumptions. In detail, the bound surface currents can be regconized as being generated by magnetization vector $\mathbf M$. There is a geometrical symmetry throughout the axis of the cylinder, and the generated magnetic field due to 
these magnetization vector is uniform, has the opposite direction as $\mathbf B_0$. Therefore, it is rational to assume the magnetization's distribution is uniform and has the opposite direction as $\mathbf{B}_0$.(See Fig. \ref{b1}) \newline
Since $\mathbf{K}_b =\mathbf M\times \hat{r}$, where $\mathbf{K}_b$ is the bound surface current density, $\mathbf{K}_b$ is just able to point to the opposite direction of $z$-axis for $\sin\theta \geq 0$, and vice versa.
Moreover, at $\theta=\pi/2$, the bound surface current density reaches its maximum, and is eliminated at $\theta=0, \pi$.(See Fig.\ref{b2})
\begin{figure}[h!]
    \centering
    \begin{subfigure}[b]{0.4\textwidth}
        \centering
        \includegraphics[width=\textwidth]{Image/screenshot_17.png}
        \caption{Uniform magnetization}
        \label{b1}
    \end{subfigure}
    \hspace{10pt} % Thay đổi khoảng cách ngang giữa hai ảnh
    \begin{subfigure}[b]{0.4\textwidth}
        \centering
        \includegraphics[width=\textwidth]{Image/screenshot_18.png}
        \caption{Surface bound current density}
        \label{b2}
    \end{subfigure}
    \footnotemark
    \caption{ Cross section of the cylinder}
\end{figure}
\footnotetext{There is a small mistake in this figure , to correct let we swap the position of the sign $+$ and the sign ~ \( \hspace{-0.4em} \bullet\)~.}



\begin{figure}[H]
    \centering
    \includegraphics[width=0.5\linewidth]{Image/screenshot_19.png}
    \caption{}
    \label{b3}
\end{figure}
Observing Fig.\ref{b2}, we notice that the current distribution described above can be regarded as equivalent to the superposition of two cylindrical bodies carrying uniform current densities per unit volume $\mathbf j$ and $-\mathbf j$, whose centers are separated by a distance $l \ll R$ (see Fig.\ref{b3}).\newline
The blue region corresponds to a current density $-\mathbf j = -j\hat{z}$ (directed into the plane of the paper), while the red region corresponds to the opposite direction. The purple region represents the zone where the current density vanishes, $\mathbf j = \mathbf 0$, with $\mathbf{O_{-}O_{+}}=\mathbf l$.

\subsection{Quantitative calculation}

At this stage, our task consists of two steps:
\begin{itemize}
    \item Determining the current density $j$.
    \item Determining the magnetic field outside the cylindrical shell.
\end{itemize}
\newpage

\begin{tcolorbox}[colback=blue!10, colframe=blue!50!black, title=Region $r \leq R$:]

We first determine the auxiliary magnetic field produced by the cylindrical body.\newline
According to Ampere's law,
\begin{align}
    &\oint \mathbf B'\cdot d\mathbf l = \mu_0 I_{\mathrm{enc}} = \mu_0 j\pi r^2.\\
    \implies\;& \mathbf B'_{\pm}= \pm\frac{\mu_0 j}{2}r_{\pm}\hat\theta_{\pm}.\\
    \implies\;& \mathbf B' = \mathbf B'_+ +\mathbf B'_- = -\frac{\mu_0 jl}{2}\hat{x}.
\end{align}

Indeed,
\begin{align}
    r_+ \hat{\theta}_+ - r_- \hat{\theta}_-
    = \hat{z}\times\mathbf{r}_+ -\hat{z}\times\mathbf{r}_-
    = -\hat{z}\times\mathbf l
    = -l\hat{x}.
\end{align}

The total magnetic field is therefore
\begin{align}
    &\mathbf B = \mathbf B_0 +\mathbf B' =\mathbf 0.\\
    \implies\;& j= \frac{2B_0}{\mu_0 l}.\\
    \implies\;& K_b = jl\sin\theta= \frac{2B_0}{\mu_0}\sin\theta.
\end{align}

As expected, $K_b$ reaches its maximum at $\theta =\frac{\pi}{2}$ and vanishes at $\theta=0$.
\end{tcolorbox}

\footnote{
The two superposed cylindrical bodies carrying opposite current densities inside the cylinder are in fact equivalent to two antiparallel currents coinciding with the principal axes of the respective cylinders, each having magnitude $I= j\pi R^2$. The term ``equivalent'' here means that they generate exactly the same magnetic field in the region outside the cylinder. Figuratively speaking, these two infinitely long currents are magnetic images obtained via the method of images.

In the electrostatic analogue of the configuration considered here—namely, a long conducting cylinder placed in a uniform electric field—one may take as images two oppositely charged infinite line charges parallel to the axis of the cylinder, which can be inferred from the induced charge distribution on its surface. However, such a construction is by no means obvious in the present magnetostatic problem, and even the inference of the direction of the magnetization vector $\mathbf M$ relies largely on physical intuition. Clearly, a separate problem is required to establish the foundational result, such as the determination of equipotential surfaces generated by a pair of oppositely charged infinite line charges in electrostatics.
}

\begin{figure}[H]
    \centering
    \includegraphics[width=0.5\linewidth]{Image/screenshot_20.png}
    \caption{Sketch illustrating the vector configuration}
    \label{}
\end{figure}

\begin{tcolorbox}[colback=blue!10, colframe=blue!50!black, title=Region $r \geq R$:]

According to Ampere's law,
\begin{align}
    &\oint \mathbf B'\cdot d\mathbf l = B' 2\pi r = \mu_0 I_{\mathrm{enc}},\\
    \text{with}~ &I_{\mathrm{enc}} = j\pi R^2.\\
    \implies\;& \mathbf {B'}_{\pm} = \pm \frac{B_0 R^2}{l r_{\pm}} \hat{\theta}_{\pm}. \\
    \implies\;& \mathbf B' = \frac{B_0 R^2}{l r_+}\hat{\theta}_+ 
    - \frac{B_0 R^2}{l r_-}\hat{\theta}_- .
\end{align}

Note that
\begin{align}
    &\frac{\hat{\theta}_{\pm}}{r_{\pm}} 
    = \hat{z}\times\frac{\mathbf{r}_{\pm}}{r_{\pm}^2},
\end{align}
where
\begin{align}
    &\mathbf{r}_{\pm} = -\left(\mathbf r \pm \frac{\mathbf l}{2}\right),
\end{align}
and therefore
\begin{align}
    &r_{\pm}^2 = r^2 + \frac{l^2}{4} \mp \mathbf r \cdot \mathbf l .
\end{align}

Under the assumption
\begin{align}
    &l \ll R \leq r,
\end{align}
we have
\begin{align}
    &\frac{1}{r_{\pm}^2} \approx \frac{1}{r^2}
    \left(1 \pm \frac{\mathbf r \cdot \mathbf l}{r^2}\right).
\end{align}

Proceeding further and neglecting terms of second order in $\frac{l}{r}$, we obtain
\begin{align}
    \frac{\hat{\theta}_+}{r_+} - \frac{\hat{\theta}_-}{r_-}
    = \frac{1}{r^2}
    \left[ 2(\mathbf r \cdot \mathbf l)\hat{\theta}
    - \hat{z}\times\mathbf l \right].
\end{align}

Hence,
\begin{align}
    \mathbf B' =
    \frac{B_0 R^2}{l r^2}
    \left[ 2(\mathbf r \cdot \mathbf l)\hat{\theta}
    - \hat{z}\times\mathbf l \right].
\end{align}

The total magnetic field is therefore
\begin{align}
    &\mathbf B = \mathbf B' + \mathbf B_0,\\
    &\mathbf B
    = B_0 \left(1-\frac{R^2}{r^2}\right)\cos\theta~\hat{r}
    - B_0 \left(1+\frac{R^2}{r^2}\right)\sin\theta~\hat{\theta}.
    \label{eq24}
\end{align}

\end{tcolorbox}


Equation~\ref{eq24} is precisely our final result, from which we may confidently assert that the solution to the main problem is
\begin{empheq}[box=\fbox]{align}
    \mathbf{v} &= v_0 \left( 1 - \frac{R^2}{r^2} \right) \cos \theta \, \hat{r} 
    - v_0 \left( 1 + \frac{R^2}{r^2} \right) \sin \theta \, \hat{\theta}.
\end{empheq}

A quick verification shows that the result obtained indeed satisfies the boundary conditions given in Eq.~\ref{eq3}.

At this point, one might question whether, in the magnetostatic problem considered earlier, choosing a superconducting half-cylinder attached to a superconducting plane would be more appropriate than a full cylinder. Indeed, in that case, the differential equations and boundary conditions of the fluid-mechanics and magnetostatic problems are formally identical throughout the entire space, rather than only in the region $y \geq 0$. Nevertheless, we may assert that the final result remains unchanged regardless of the chosen configuration. This follows from the symmetry of the system; more precisely, Eq.~\ref{eq24} shows that the magnetic field possesses only a tangential component with respect to the plane $(\Sigma): y=0$. Consequently, the superconducting plane merely induces surface currents that cancel this tangential component, without contributing any magnetic field in the region $y \geq 0$.

\newpage
\section{Velocity superposition}\label{III}
The superposition principle, inspired by the correspondence principle in electromagnetism, allows us to decompose a complex problem into simpler ones and thereby construct a complicated field from more elementary models.

The analogous linearity of the differential equations governing fluid flow enables us to proceed in exactly the same manner.

In this framework, for each distinct flow characterized by velocity fields $\mathbf{v}_0, \mathbf{v}_1, \mathbf{v}_2, \ldots$, the flow satisfying the complete problem is described by the velocity field
\begin{empheq}[box=\fbox]{align}
    \mathbf{v} = \mathbf{v}_0 + \mathbf{v}_1 + \mathbf{v}_2 + \ldots.
\end{empheq}

We shall now examine several fields that are directly related to the problem at hand, namely incompressible and irrotational (potential) flows.

\subsection{Electrostatic analogy}
Equation~\ref{eq1} asserts that there exists a scalar quantity $\phi$ associated with the velocity field such that
\[
\mathbf{v} = \nabla \phi,
\]
since
\[
\nabla \times \bigl(\nabla f(x,y,z)\bigr) = \mathbf{0}
\]
for any scalar function $f(x,y,z)$. \newline

The scalar $\phi$ is therefore called the \emph{velocity potential}. Moreover, equation~\ref{eq2} implies that
\[
\nabla \cdot \bigl(\nabla \phi\bigr) = 0,
\]
or equivalently,
\begin{align}
\nabla^2 \phi = 0.
\label{eq55}
\end{align}

We now present a comparison table:
\vspace{5mm}

\begin{center}
\begin{tabular}{|>{\centering\arraybackslash}m{5.5cm}|>{\centering\arraybackslash}m{5.5cm}|}
    \hline
    \textbf{Electrostatic field in a charge-free region} &
    \textbf{Velocity field of a fluid in potential flow} \\
    \hline
    $\nabla \times \mathbf{E} = \mathbf{0}$, hence there exists a scalar potential $V$ such that
    $\mathbf{E} = -\nabla V$. &
    $\nabla \times \mathbf{v} = \mathbf{0}$, hence there exists a scalar potential $\phi$ such that
    $\mathbf{v} = \nabla \phi$. \\
    \hline
    $\mathbf{E}$ is orthogonal to the surfaces $V = \mathrm{const}$. &
    $\mathbf{v}$ is orthogonal to the surfaces $\phi = \mathrm{const}$. \\
    \hline
    $\nabla \cdot \mathbf{E} = 0$, or equivalently $\nabla^2 V = 0$. &
    $\nabla \cdot \mathbf{v} = 0$, or equivalently $\nabla^2 \phi = 0$. \\
    \hline
\end{tabular}
\end{center}


\subsection{Two-dimensional sources and sinks}
Consider an infinitely long straight line charge, uniformly charged with linear charge density $\lambda$.  
The cylindrical symmetry of the problem implies an electric field of the form
\[
\mathbf{E} = E(r)\,\hat{r}.
\]

This field is defined everywhere in space except on the wire itself. In this region there is no charge, and therefore
\[
\nabla \cdot \mathbf{E} = 0.
\]
Applying Gauss’s law to a cylindrical surface of radius $r$ and height $h$, we obtain
\begin{align}
    \Phi
    = \iint_{\mathbf{S}} \mathbf{E} \cdot \hat{n}\, dS
    = \frac{\lambda h}{\varepsilon_0}
    = \iiint_{\mathbf{V}} \nabla \cdot \mathbf{E}\, dV
    \neq 0 .
\end{align}

This apparent contradiction arises because the field is not defined at $r=0$. This property may be described by the Dirac delta distribution $\delta^3(\mathbf{r})$, although it is unnecessary to invoke it here, since we are only concerned with the field in space, where it is well defined and given by
\[
E(r) = \frac{\lambda}{2\pi \varepsilon_0 r}.
\]

Let us now examine a potential flow of a fluid with analogous dynamics, namely a velocity field of the form
\[
\mathbf{v} = \frac{k}{r}\,\hat{r}.
\]

The fluid velocity vanishes at infinity relative to the $(Oz)$ axis, and the flux of the field is conserved through any cylindrical surface of height $h$ and radius $r$:
\begin{align}
    \Phi = 2\pi r h\, v(r) = 2\pi h k.
\end{align}

This nonzero flux contradicts $\nabla \cdot \mathbf{v} = 0$ for the same reason as above, and the axis $(Oz)$ constitutes a set of singular points. Indeed, just as the charged wire acts as the source of the electrostatic field, the axis $(Oz)$ is the origin of the flow under consideration: it must be regarded as either emitting or absorbing fluid.

The quantity $\Phi/h$, namely the flux per unit length, represents the volumetric flow rate of the fluid per unit length of the source axis (the amount of fluid crossing a closed surface per unit length per unit time). Denoting this quantity by $\Lambda$, it plays a role analogous to that of the linear charge density $\lambda$ in the electrostatic model. One then has
\begin{align}
    \mathbf{v} = \frac{\Lambda}{2\pi r}\,\hat{r}.
\end{align}

Depending on whether $\Lambda$ is positive or negative, the axis $(Oz)$ is interpreted as a two-dimensional \emph{source} or \emph{sink}. A perforated irrigation pipe with a large number of uniformly distributed small holes provides an intuitive physical picture of the model under study.

It is readily seen that the velocity potential may be defined such that
\[
d\phi = \mathbf{v} \cdot d\mathbf{l},
\]
which yields
\begin{align}
    \phi = \frac{\Lambda}{2\pi}\ln\!\left(\frac{r}{a}\right),
    \qquad a = \text{const}.
    \label{eq23}
\end{align}


\subsection{Hydrodynamic dipole}
Two-dimensional sources and sinks naturally evoke the notion of ``monopoles''.  
We now examine the configuration obtained by placing a source and a sink close to each other.
Figure~\ref{c1} shows the resulting streamlines of the flow, while Fig.~\ref{c2} displays the electric field lines of an electric dipole.

\begin{figure}[H]
    \centering
    \begin{subfigure}[b]{0.4\textwidth}
        \centering
        \includegraphics[width=\textwidth]{Image/luong cuc thuy dong luc.jpg}
        \caption{}
        \label{c1}
    \end{subfigure}
    \hspace{10pt}
    \begin{subfigure}[b]{0.4\textwidth}
        \centering
        \includegraphics[width=\textwidth]{Image/luong cuc dien.jpg}
        \caption{}
        \label{c2}
    \end{subfigure}
    \caption{}
\end{figure}

\paragraph{Remarks.}
\begin{itemize}
    \item The streamlines may be circular arcs.
    \item Although the two figures are visually similar, the electric field lines of an electric dipole are not circular.
\end{itemize}

We now investigate the phenomenon quantitatively by introducing the coordinate system shown below:
\begin{figure}[H]
    \centering
    \includegraphics[width=0.4\textwidth]{Image/screenshot_1766923724.png}
    \caption{Schematic for the coordinates of a source–sink pair}
\end{figure}
\begin{tcolorbox}[colback=blue!10, colframe=blue!50!black, title= ]
Using the result of Eq.~\eqref{eq23}, the velocity potential at a point $M(r,\theta)$ is given by
\begin{align}
    \phi = \frac{\Lambda}{2\pi}\ln\!\left(\frac{r_+}{r_-}\right).
\end{align}

Expressing this in terms of $r$ and $\theta$, we have
\begin{itemize}
    \item $r_+ = \sqrt{r^2 + d^2 - 2rd\cos\theta}$,
    \item $r_- = \sqrt{r^2 + d^2 + 2rd\cos\theta}$.
\end{itemize}

Placing the source and the sink close to each other corresponds to the condition $d \ll r$.
We therefore perform a first-order approximation:
\begin{itemize}
    \item $r_+ \approx r\left(1 - \frac{d\cos\theta}{r}\right)$,
    \item $r_- \approx r\left(1 + \frac{d\cos\theta}{r}\right)$.
\end{itemize}

This yields
\begin{align}
    \phi
    &= \frac{\Lambda}{2\pi}
       \ln\!\left(
       \frac{1-\frac{d\cos\theta}{r}}{1+\frac{d\cos\theta}{r}}
       \right)
       \approx
       \frac{\Lambda}{2\pi}
       \ln\!\left(1-\frac{d\cos\theta}{r}\right)^2, \\
    \implies
    \phi
    &\approx
    -\frac{\Lambda}{2\pi}\frac{2d\cos\theta}{r}.
\end{align}

This expression may be rewritten in the familiar form
\begin{align}
    \phi = \frac{\mathbf{p}\cdot\hat{\mathbf{r}}}{2\pi r},
\end{align}
where the dipole strength is defined as $\mathbf{p} = 2d\Lambda\,\hat{\mathbf{z}}$.

From this, the velocity field follows immediately:
\begin{align}
    \mathbf{v}
    = \nabla\phi
    = \frac{p}{2\pi r^2}
      \left(
      \cos\theta\,\hat{\mathbf{r}}
      + \sin\theta\,\hat{\boldsymbol{\theta}}
      \right).
\end{align}
\end{tcolorbox}

Such a closely spaced source–sink pair is called a \emph{hydrodynamic dipole}.


\subsection{Flow around a cylindrical obstacle}
In this section, we rederive the result~\eqref{eq24} using the principle of velocity superposition.

We first recall the boundary conditions~\ref{eq3} presented in Section~\ref{I}. \newline
The first condition states that far from the obstacle, the flow must be uniform. This suggests interpreting the total velocity field as the superposition of a uniform flow and an auxiliary field $\mathbf{v}'$, which must vanish as $r \to \infty$ for all values of the polar angle $\theta$. \newline
At the same time, the second boundary condition implies that
\[
v_r(R,\theta) = v_0 \cos\theta + v'_r(R,\theta) = 0.
\]
Hence,
\[
v'_r(R,\theta) = -v_0 \cos\theta.
\]

A velocity field satisfying this condition was encountered previously, namely the field of a hydrodynamic dipole:
\begin{align}
    v_r = \frac{p}{2\pi r^2}\cos\theta.
\end{align}

Applying the boundary condition at the surface of the cylinder $r = R$, we obtain
\begin{align}
    \frac{p}{2\pi R^2}\cos\theta = -v_0 \cos\theta,
\end{align}
which immediately yields
\begin{align}
    p = -2\pi R^2 v_0.
\end{align}

The total velocity field in polar coordinates, written as the superposition
\[
\mathbf{v} = \mathbf{v}_0 + \mathbf{v}',
\]
is therefore given by
\begin{align}
    \mathbf{v}
    = v_0\left(1-\frac{R^2}{r^2}\right)\cos\theta~\hat{\mathbf{r}}
    - v_0\left(1+\frac{R^2}{r^2}\right)\sin\theta~\hat{\boldsymbol{\theta}}.
\end{align}

Thus, equation~\eqref{eq24} is recovered.


\newpage
\section{Laplace's equation}\label{IV}
\subsection{The big picture}

Equation~\eqref{eq55}, which we have obtained, is known as a \emph{Laplace equation}. Together with the boundary conditions of the problem, it provides sufficient information to describe a physical phenomenon associated with a time-independent potential field (i.e.\ in the steady state), such as the electrostatic field or the velocity field of a fluid.

This type of modeling appears very broadly in physics; we list a few representative examples below:
\renewcommand{\labelitemii}{*}

\begin{itemize}
    \item Particle diffusion obeying Fick's law:
    \begin{itemize}
        \item Scalar field: particle density $n$;
        \item Vector field: $\mathbf{j}_D = -D\nabla n$, where $D$ is the diffusion coefficient;
    \end{itemize}
    \item Heat diffusion obeying Fourier's law:
    \begin{itemize}
        \item Scalar field: temperature $T$;
        \item Vector field: $\mathbf{j}_k = -k\nabla T$, where $k$ is the thermal conductivity;
    \end{itemize}
    \item Electrical conduction obeying Ohm's law:
    \begin{itemize}
        \item Scalar field: electric potential $V$;
        \item Vector field: $\mathbf{j} = \sigma\nabla V$, where $\sigma$ is the electrical conductivity;
    \end{itemize}
\end{itemize}

In general, one considers a scalar function $\zeta$ satisfying the Laplace equation
\[
\nabla^2 \zeta = 0,
\]
and a vector field $\mathbf{J}$, called the flux, related to $\zeta$ by
\[
\mathbf{J} = \Upsilon \nabla \zeta,
\]
where $\Upsilon$ is a constant characteristic of the physical problem.

These properties, together with the geometric boundary conditions and imposed limits specific to each problem, imply the uniqueness of the solution to the Laplace equation $\nabla^2 \zeta = 0$ (by the uniqueness theorem).

Consequently, two problems associated with identical or analogous physical quantities, in the sense described above, and sharing the same configuration, will possess identical or analogous solutions, uniquely determined.

\subsection{General solution in cylindrical coordinates}

We now rederive the general solution of equation~\eqref{eq55} in cylindrical coordinates, which is simultaneously the general solution of any Laplace equation possessing cylindrical symmetry.

\begin{tcolorbox}[colback=blue!10, colframe=blue!50!black, title= ]
We first seek a particular solution of the form
\[
\phi = S(r)\Theta(\theta).
\]
Substituting into the Laplace operator yields
\begin{align}
    \nabla^2 \phi
    &= \frac{1}{r}\frac{\partial}{\partial r}\left(r\frac{\partial\phi}{\partial r}\right)
    + \frac{1}{r^2}\frac{\partial^2 \phi}{\partial\theta^2}=0,\\
    \implies\;
    &\frac{r}{S}\frac{d}{dr}\left(r\frac{dS}{dr}\right)
    + \frac{1}{\Theta}\frac{d^2\Theta}{d\theta^2} = 0.
\end{align}

Since $S$ depends only on $r$ and $\Theta$ depends only on $\theta$, each term must be equal to a constant:
\begin{align}
    \frac{r}{S}\frac{d}{dr}\left(r\frac{dS}{dr}\right) &= C_1,\\
    \frac{1}{\Theta}\frac{d^2\Theta}{d\theta^2} &= C_2, \label{eq4.4}\\
    C_1 + C_2 &= 0.
\end{align}

We observe that $C_2$ must be negative, since $\Theta(\theta)$ must be a periodic function. If $C_2$ were positive, $\Theta$ would be exponential in $\theta$, which is unphysical. We therefore set $C_2 = -k^2$, obtaining
\begin{align}
    \Theta(\theta) = A\cos k\theta + B\sin k\theta.
\end{align}

Furthermore, the condition $\Theta(\theta+2\pi)=\Theta(\theta)$ requires $k$ to be an integer:
\[
k = 0,1,2,3,\dots
\]

The radial equation
\[
\frac{r}{S}\frac{d}{dr}\left(r\frac{dS}{dr}\right)=k^2
\]
admits power-law solutions $S=r^n$, where $n$ satisfies
\[
r\frac{d}{dr}(rn r^{n-1}) = n^2 r^n = k^2 S,
\quad\Rightarrow\quad n=\pm k.
\]

Thus, the general solution for $k\neq 0$ is
\begin{align}
    S(r)= C r^k + D r^{-k}.
\end{align}

The case $k=0$ must be treated separately. Substituting $k=0$ gives
\begin{align}
    \frac{r}{S}\frac{d}{dr}\left(r\frac{dS}{dr}\right)=0
    \;\Rightarrow\;
    r\frac{dS}{dr}=C
    \;\Rightarrow\;
    S=C\ln r + D.
\end{align}

Substituting $k=0$ into~\eqref{eq4.4} yields
\[
\frac{d^2\Theta}{d\theta^2}=0
\;\Rightarrow\;
\Theta=A\theta+B,
\]
which is non-periodic and therefore unphysical.

The general solution is obtained as a linear superposition of all admissible solutions:
\begin{align}
    \phi(r,\theta)
    = a_0 + b_0 \ln r
    + \sum_{k=1}^{\infty}
    \Big[
    r^k\big(a_k\cos k\theta + b_k\sin k\theta\big)
    + r^{-k}\big(c_k\cos k\theta + d_k\sin k\theta\big)
    \Big].
    \label{eq57}
\end{align}
\end{tcolorbox}

\subsection{A revision}

Imposing $\phi=0$ at $r=0$, the boundary conditions~\ref{eq3} may be rewritten as
\begin{align}
    \left\{
    \begin{array}{ll}
        \displaystyle\lim_{r\to\infty}\phi = v_0 r\cos\theta, \\[12pt]
        \displaystyle\frac{d\phi}{dr}\bigg|_{r=R} = 0.
    \end{array}
    \right.
\end{align}

The first condition can only be satisfied if
\[
a_0=b_0=b_k=d_k=0,
\]
and since $\theta$ is arbitrary, one must have $k=1$.

Thus,
\[
\phi = v_0\cos\theta\left(r+\frac{d_1}{r}\right).
\]

Applying the second boundary condition yields
\[
d_1 = v_0 R^2.
\]

Hence, the velocity potential is
\begin{align}
    \phi(r,\theta)=v_0\cos\theta\left(r+\frac{R^2}{r}\right).
\end{align}

The velocity components follow as
\begin{align}
    v_r &= \frac{d\phi}{dr}
    = v_0\cos\theta\left(1-\frac{R^2}{r^2}\right),\\
    v_\theta &= \frac{1}{r}\frac{d\phi}{d\theta}
    = -v_0\sin\theta\left(1+\frac{R^2}{r^2}\right).
\end{align}

Therefore,
\begin{empheq}[box=\fbox]{align*}
    \mathbf{v}
    &= v_0\left(1-\frac{R^2}{r^2}\right)\cos\theta~\hat{\mathbf{r}}
    - v_0\left(1+\frac{R^2}{r^2}\right)\sin\theta~\hat{\boldsymbol{\theta}}.
\end{empheq}

\end{document}