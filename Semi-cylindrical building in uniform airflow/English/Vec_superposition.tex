The superposition principle, inspired by the correspondence principle in electromagnetism, allows us to decompose a complex problem into simpler ones and thereby construct a complicated field from more elementary models.

The analogous linearity of the differential equations governing fluid flow enables us to proceed in exactly the same manner.

In this framework, for each distinct flow characterized by velocity fields $\mathbf{v}_0, \mathbf{v}_1, \mathbf{v}_2, \ldots$, the flow satisfying the complete problem is described by the velocity field
\begin{empheq}[box=\fbox]{align}
    \mathbf{v} = \mathbf{v}_0 + \mathbf{v}_1 + \mathbf{v}_2 + \ldots.
\end{empheq}

We shall now examine several fields that are directly related to the problem at hand, namely incompressible and irrotational (potential) flows.

\subsection{Electrostatic analogy}
Equation~\ref{eq1} asserts that there exists a scalar quantity $\phi$ associated with the velocity field such that
\[
\mathbf{v} = \nabla \phi,
\]
since
\[
\nabla \times \bigl(\nabla f(x,y,z)\bigr) = \mathbf{0}
\]
for any scalar function $f(x,y,z)$. \newline

The scalar $\phi$ is therefore called the \emph{velocity potential}. Moreover, equation~\ref{eq2} implies that
\[
\nabla \cdot \bigl(\nabla \phi\bigr) = 0,
\]
or equivalently,
\begin{align}
\nabla^2 \phi = 0.
\label{eq55}
\end{align}

We now present a comparison table:
\vspace{5mm}

\begin{center}
\begin{tabular}{|>{\centering\arraybackslash}m{5.5cm}|>{\centering\arraybackslash}m{5.5cm}|}
    \hline
    \textbf{Electrostatic field in a charge-free region} &
    \textbf{Velocity field of a fluid in potential flow} \\
    \hline
    $\nabla \times \mathbf{E} = \mathbf{0}$, hence there exists a scalar potential $V$ such that
    $\mathbf{E} = -\nabla V$. &
    $\nabla \times \mathbf{v} = \mathbf{0}$, hence there exists a scalar potential $\phi$ such that
    $\mathbf{v} = \nabla \phi$. \\
    \hline
    $\mathbf{E}$ is orthogonal to the surfaces $V = \mathrm{const}$. &
    $\mathbf{v}$ is orthogonal to the surfaces $\phi = \mathrm{const}$. \\
    \hline
    $\nabla \cdot \mathbf{E} = 0$, or equivalently $\nabla^2 V = 0$. &
    $\nabla \cdot \mathbf{v} = 0$, or equivalently $\nabla^2 \phi = 0$. \\
    \hline
\end{tabular}
\end{center}


\subsection{Two-dimensional sources and sinks}
Consider an infinitely long straight line charge, uniformly charged with linear charge density $\lambda$.  
The cylindrical symmetry of the problem implies an electric field of the form
\[
\mathbf{E} = E(r)\,\hat{r}.
\]

This field is defined everywhere in space except on the wire itself. In this region there is no charge, and therefore
\[
\nabla \cdot \mathbf{E} = 0.
\]
Applying Gauss’s law to a cylindrical surface of radius $r$ and height $h$, we obtain
\begin{align}
    \Phi
    = \iint_{\mathbf{S}} \mathbf{E} \cdot \hat{n}\, dS
    = \frac{\lambda h}{\varepsilon_0}
    = \iiint_{\mathbf{V}} \nabla \cdot \mathbf{E}\, dV
    \neq 0 .
\end{align}

This apparent contradiction arises because the field is not defined at $r=0$. This property may be described by the Dirac delta distribution $\delta^3(\mathbf{r})$, although it is unnecessary to invoke it here, since we are only concerned with the field in space, where it is well defined and given by
\[
E(r) = \frac{\lambda}{2\pi \varepsilon_0 r}.
\]

Let us now examine a potential flow of a fluid with analogous dynamics, namely a velocity field of the form
\[
\mathbf{v} = \frac{k}{r}\,\hat{r}.
\]

The fluid velocity vanishes at infinity relative to the $(Oz)$ axis, and the flux of the field is conserved through any cylindrical surface of height $h$ and radius $r$:
\begin{align}
    \Phi = 2\pi r h\, v(r) = 2\pi h k.
\end{align}

This nonzero flux contradicts $\nabla \cdot \mathbf{v} = 0$ for the same reason as above, and the axis $(Oz)$ constitutes a set of singular points. Indeed, just as the charged wire acts as the source of the electrostatic field, the axis $(Oz)$ is the origin of the flow under consideration: it must be regarded as either emitting or absorbing fluid.

The quantity $\Phi/h$, namely the flux per unit length, represents the volumetric flow rate of the fluid per unit length of the source axis (the amount of fluid crossing a closed surface per unit length per unit time). Denoting this quantity by $\Lambda$, it plays a role analogous to that of the linear charge density $\lambda$ in the electrostatic model. One then has
\begin{align}
    \mathbf{v} = \frac{\Lambda}{2\pi r}\,\hat{r}.
\end{align}

Depending on whether $\Lambda$ is positive or negative, the axis $(Oz)$ is interpreted as a two-dimensional \emph{source} or \emph{sink}. A perforated irrigation pipe with a large number of uniformly distributed small holes provides an intuitive physical picture of the model under study.

It is readily seen that the velocity potential may be defined such that
\[
d\phi = \mathbf{v} \cdot d\mathbf{l},
\]
which yields
\begin{align}
    \phi = \frac{\Lambda}{2\pi}\ln\!\left(\frac{r}{a}\right),
    \qquad a = \text{const}.
    \label{eq23}
\end{align}


\subsection{Hydrodynamic dipole}
Two-dimensional sources and sinks naturally evoke the notion of ``monopoles''.  
We now examine the configuration obtained by placing a source and a sink close to each other.
Figure~\ref{c1} shows the resulting streamlines of the flow, while Fig.~\ref{c2} displays the electric field lines of an electric dipole.

\begin{figure}[H]
    \centering
    \begin{subfigure}[b]{0.4\textwidth}
        \centering
        \includegraphics[width=\textwidth]{Image/luong cuc thuy dong luc.jpg}
        \caption{}
        \label{c1}
    \end{subfigure}
    \hspace{10pt}
    \begin{subfigure}[b]{0.4\textwidth}
        \centering
        \includegraphics[width=\textwidth]{Image/luong cuc dien.jpg}
        \caption{}
        \label{c2}
    \end{subfigure}
    \caption{}
\end{figure}

\paragraph{Remarks.}
\begin{itemize}
    \item The streamlines may be circular arcs.
    \item Although the two figures are visually similar, the electric field lines of an electric dipole are not circular.
\end{itemize}

We now investigate the phenomenon quantitatively by introducing the coordinate system shown below:
\begin{figure}[H]
    \centering
    \includegraphics[width=0.4\textwidth]{Image/screenshot_1766923724.png}
    \caption{Schematic for the coordinates of a source–sink pair}
\end{figure}
\begin{tcolorbox}[colback=blue!10, colframe=blue!50!black, title= ]
Using the result of Eq.~\eqref{eq23}, the velocity potential at a point $M(r,\theta)$ is given by
\begin{align}
    \phi = \frac{\Lambda}{2\pi}\ln\!\left(\frac{r_+}{r_-}\right).
\end{align}

Expressing this in terms of $r$ and $\theta$, we have
\begin{itemize}
    \item $r_+ = \sqrt{r^2 + d^2 - 2rd\cos\theta}$,
    \item $r_- = \sqrt{r^2 + d^2 + 2rd\cos\theta}$.
\end{itemize}

Placing the source and the sink close to each other corresponds to the condition $d \ll r$.
We therefore perform a first-order approximation:
\begin{itemize}
    \item $r_+ \approx r\left(1 - \frac{d\cos\theta}{r}\right)$,
    \item $r_- \approx r\left(1 + \frac{d\cos\theta}{r}\right)$.
\end{itemize}

This yields
\begin{align}
    \phi
    &= \frac{\Lambda}{2\pi}
       \ln\!\left(
       \frac{1-\frac{d\cos\theta}{r}}{1+\frac{d\cos\theta}{r}}
       \right)
       \approx
       \frac{\Lambda}{2\pi}
       \ln\!\left(1-\frac{d\cos\theta}{r}\right)^2, \\
    \implies
    \phi
    &\approx
    -\frac{\Lambda}{2\pi}\frac{2d\cos\theta}{r}.
\end{align}

This expression may be rewritten in the familiar form
\begin{align}
    \phi = \frac{\mathbf{p}\cdot\hat{\mathbf{r}}}{2\pi r},
\end{align}
where the dipole strength is defined as $\mathbf{p} = 2d\Lambda\,\hat{\mathbf{z}}$.

From this, the velocity field follows immediately:
\begin{align}
    \mathbf{v}
    = \nabla\phi
    = \frac{p}{2\pi r^2}
      \left(
      \cos\theta\,\hat{\mathbf{r}}
      + \sin\theta\,\hat{\boldsymbol{\theta}}
      \right).
\end{align}
\end{tcolorbox}

Such a closely spaced source–sink pair is called a \emph{hydrodynamic dipole}.


\subsection{Flow around a cylindrical obstacle}
In this section, we rederive the result~\eqref{eq24} using the principle of velocity superposition.

We first recall the boundary conditions~\ref{eq3} presented in Section~\ref{I}. \newline
The first condition states that far from the obstacle, the flow must be uniform. This suggests interpreting the total velocity field as the superposition of a uniform flow and an auxiliary field $\mathbf{v}'$, which must vanish as $r \to \infty$ for all values of the polar angle $\theta$. \newline
At the same time, the second boundary condition implies that
\[
v_r(R,\theta) = v_0 \cos\theta + v'_r(R,\theta) = 0.
\]
Hence,
\[
v'_r(R,\theta) = -v_0 \cos\theta.
\]

A velocity field satisfying this condition was encountered previously, namely the field of a hydrodynamic dipole:
\begin{align}
    v_r = \frac{p}{2\pi r^2}\cos\theta.
\end{align}

Applying the boundary condition at the surface of the cylinder $r = R$, we obtain
\begin{align}
    \frac{p}{2\pi R^2}\cos\theta = -v_0 \cos\theta,
\end{align}
which immediately yields
\begin{align}
    p = -2\pi R^2 v_0.
\end{align}

The total velocity field in polar coordinates, written as the superposition
\[
\mathbf{v} = \mathbf{v}_0 + \mathbf{v}',
\]
is therefore given by
\begin{align}
    \mathbf{v}
    = v_0\left(1-\frac{R^2}{r^2}\right)\cos\theta~\hat{\mathbf{r}}
    - v_0\left(1+\frac{R^2}{r^2}\right)\sin\theta~\hat{\boldsymbol{\theta}}.
\end{align}

Thus, equation~\eqref{eq24} is recovered.

