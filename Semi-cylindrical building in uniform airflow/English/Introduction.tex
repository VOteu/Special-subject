\begin{figure}[H]
    \centering
    \includegraphics[width=0.8\linewidth]{Image/problem.png}
    \caption{Schematic of a long semi-cylindrical obstacle in uniform airflow}
    \label{F2}
\end{figure}
\begin{tcolorbox}[colback=blue!10, colframe=blue!50!black, title= Problem Statement]
Consider a building in the shape of a semi-cylinder of length $L$ and radius $R$, resting on an infinite flat ground, such that the length is very large compared with the radius (See \ref{F2}). From infinity, a stream of wind blows toward it with velocity $\mathbf{v}_0$. Determine the wind velocity at every point in space once the flow has reached a steady state.
\end{tcolorbox}
The crucial assumptions are that the airflow is incompressible and irrotational (with a very small vicosity coefficient). This means that the velocity field $\mathbf{v}$ satisfies the following equations, in region of $y\geq 0$: \begin{align}
    &\nabla\cdot\mathbf v = \mathbf 0, \label{eq1}\\
    &\nabla\times\mathbf v=\mathbf 0. \label{eq2}
\end{align} And the boundary conditions are: \begin{align}
    &\left\{
    \begin{array}{ll}
        \displaystyle \lim_{r \to \infty} \mathbf{v}(r, \theta) = \mathbf{v}_0, \\[12pt] 
        \mathbf{v}(R, \theta)\cdot\hat{r} = \mathbf 0.
    \end{array}
    \right.\label{eq3}
\end{align}
In the general case, equation~\eqref{eq2} takes the form
\[
\nabla \times \mathbf{v} = 2 \mathbf{\Omega},
\]
and one readily observes the analogy between the two Maxwell equations\footnote{
\begin{align*}
\nabla \cdot \mathbf{B} &=\mathbf 0,\\
\nabla \times \mathbf{B} &= \mu_0 \mathbf{j}.
\end{align*}}
in magnetostatics and the two partial differential equations describing an ideal fluid flow.

Therefore, we shall first examine the analogous magnetostatic problem using a simplified model presented in Section~\ref{II}. We then return to the fluid-mechanics via the corresponding concepts developed in electrostatics in Section~\ref{III}. Finally, we conclude by solving the Laplace equations to verify the accuracy of the simplified models employed, while at the same time providing the most general estimation possible.
