\subsection{The equivalent model}
A perfect conductor is a conductor with an infinite conductivity. This requires that the electric field inside its body is zero. A direct consequence of this property is that the magnetic field inside the object needs to be constant. 
A superconducting material is a perfect conductor with zero magnetic permeability, i.e. $\mu = 0$. 
This means that the magnetic field inside a superconductor is always zero.
 Hence, every current distribution only can reside outside or on the surface of a superconductor, 
 and the perpendicular component of the magnetic field on the object's surface is eliminated (which is called Meissner's effect).
\begin{figure}[H]
    \centering
    \includegraphics[width=0.5\textwidth]{Image/4fieldphysic.jpg}
    \caption{Meissner's effect}
    \label{F3}
\end{figure}
Considering a superconducting cylinder of infinite length, radius $R$ in the presence of a uniform magnetic field $\mathbf{B}_0$ directed perpendicularly to its axis.
When the steady state is reached, on the surface of the cyliner emerges bound surface currents generating a supplementary magnetic field equivalent to $-\mathbf{B}_0$, so that the net magnetic field is zero, inside the object and changing the field outside.
Differential equations and boundary conditions describing the system in this state are:
\begin{align}
    &\left\{
    \begin{array}{ll}
        \nabla\cdot\mathbf B=\mathbf 0, \\[12pt] 
        \nabla\times\mathbf{B} = \mathbf 0,
    \end{array}
    \right.
    &\left\{
    \begin{array}{ll}
        \displaystyle \lim_{r \to \infty} \mathbf {B}(r, \theta) = \mathbf{B}_0, \\[12pt] 
        \mathbf{B}(R, \theta)\cdot\hat{r} = \mathbf 0.
    \end{array}
    \right.\label{eq4}
\end{align} 
Obviously, the solution of this problem must be similar to that of the previous problem in their form cause their governing equations and boundary conditions are identical in forms. \newline
To continue, we will offer some guesses and assumptions. In detail, the bound surface currents can be regconized as being generated by magnetization vector $\mathbf M$. There is a geometrical symmetry throughout the axis of the cylinder, and the generated magnetic field due to 
these magnetization vector is uniform, has the opposite direction as $\mathbf B_0$. Therefore, it is rational to assume the magnetization's distribution is uniform and has the opposite direction as $\mathbf{B}_0$.(See Fig. \ref{b1}) \newline
Since $\mathbf{K}_b =\mathbf M\times \hat{r}$, where $\mathbf{K}_b$ is the bound surface current density, $\mathbf{K}_b$ is just able to point to the opposite direction of $z$-axis for $\sin\theta \geq 0$, and vice versa.
Moreover, at $\theta=\pi/2$, the bound surface current density reaches its maximum, and is eliminated at $\theta=0, \pi$.(See Fig.\ref{b2})
\begin{figure}[h!]
    \centering
    \begin{subfigure}[b]{0.4\textwidth}
        \centering
        \includegraphics[width=\textwidth]{Image/screenshot_17.png}
        \caption{Uniform magnetization}
        \label{b1}
    \end{subfigure}
    \hspace{10pt} % Thay đổi khoảng cách ngang giữa hai ảnh
    \begin{subfigure}[b]{0.4\textwidth}
        \centering
        \includegraphics[width=\textwidth]{Image/screenshot_18.png}
        \caption{Surface bound current density}
        \label{b2}
    \end{subfigure}
    \footnotemark
    \caption{ Cross section of the cylinder}
\end{figure}
\footnotetext{There is a small mistake in this figure , to correct let we swap the position of the sign $+$ and the sign ~ \( \hspace{-0.4em} \bullet\)~.}



\begin{figure}[H]
    \centering
    \includegraphics[width=0.5\linewidth]{Image/screenshot_19.png}
    \caption{}
    \label{b3}
\end{figure}
Observing Fig.\ref{b2}, we notice that the current distribution described above can be regarded as equivalent to the superposition of two cylindrical bodies carrying uniform current densities per unit volume $\mathbf j$ and $-\mathbf j$, whose centers are separated by a distance $l \ll R$ (see Fig.\ref{b3}).\newline
The blue region corresponds to a current density $-\mathbf j = -j\hat{z}$ (directed into the plane of the paper), while the red region corresponds to the opposite direction. The purple region represents the zone where the current density vanishes, $\mathbf j = \mathbf 0$, with $\mathbf{O_{-}O_{+}}=\mathbf l$.

\subsection{Quantitative calculation}

At this stage, our task consists of two steps:
\begin{itemize}
    \item Determining the current density $j$.
    \item Determining the magnetic field outside the cylindrical shell.
\end{itemize}
\newpage

\begin{tcolorbox}[colback=blue!10, colframe=blue!50!black, title=Region $r \leq R$:]

We first determine the auxiliary magnetic field produced by the cylindrical body.\newline
According to Ampere's law,
\begin{align}
    &\oint \mathbf B'\cdot d\mathbf l = \mu_0 I_{\mathrm{enc}} = \mu_0 j\pi r^2.\\
    \implies\;& \mathbf B'_{\pm}= \pm\frac{\mu_0 j}{2}r_{\pm}\hat\theta_{\pm}.\\
    \implies\;& \mathbf B' = \mathbf B'_+ +\mathbf B'_- = -\frac{\mu_0 jl}{2}\hat{x}.
\end{align}

Indeed,
\begin{align}
    r_+ \hat{\theta}_+ - r_- \hat{\theta}_-
    = \hat{z}\times\mathbf{r}_+ -\hat{z}\times\mathbf{r}_-
    = -\hat{z}\times\mathbf l
    = -l\hat{x}.
\end{align}

The total magnetic field is therefore
\begin{align}
    &\mathbf B = \mathbf B_0 +\mathbf B' =\mathbf 0.\\
    \implies\;& j= \frac{2B_0}{\mu_0 l}.\\
    \implies\;& K_b = jl\sin\theta= \frac{2B_0}{\mu_0}\sin\theta.
\end{align}

As expected, $K_b$ reaches its maximum at $\theta =\frac{\pi}{2}$ and vanishes at $\theta=0$.
\end{tcolorbox}

\footnote{
The two superposed cylindrical bodies carrying opposite current densities inside the cylinder are in fact equivalent to two antiparallel currents coinciding with the principal axes of the respective cylinders, each having magnitude $I= j\pi R^2$. The term ``equivalent'' here means that they generate exactly the same magnetic field in the region outside the cylinder. Figuratively speaking, these two infinitely long currents are magnetic images obtained via the method of images.

In the electrostatic analogue of the configuration considered here—namely, a long conducting cylinder placed in a uniform electric field—one may take as images two oppositely charged infinite line charges parallel to the axis of the cylinder, which can be inferred from the induced charge distribution on its surface. However, such a construction is by no means obvious in the present magnetostatic problem, and even the inference of the direction of the magnetization vector $\mathbf M$ relies largely on physical intuition. Clearly, a separate problem is required to establish the foundational result, such as the determination of equipotential surfaces generated by a pair of oppositely charged infinite line charges in electrostatics.
}

\begin{figure}[H]
    \centering
    \includegraphics[width=0.5\linewidth]{Image/screenshot_20.png}
    \caption{Sketch illustrating the vector configuration}
    \label{}
\end{figure}

\begin{tcolorbox}[colback=blue!10, colframe=blue!50!black, title=Region $r \geq R$:]

According to Ampere's law,
\begin{align}
    &\oint \mathbf B'\cdot d\mathbf l = B' 2\pi r = \mu_0 I_{\mathrm{enc}},\\
    \text{with}~ &I_{\mathrm{enc}} = j\pi R^2.\\
    \implies\;& \mathbf {B'}_{\pm} = \pm \frac{B_0 R^2}{l r_{\pm}} \hat{\theta}_{\pm}. \\
    \implies\;& \mathbf B' = \frac{B_0 R^2}{l r_+}\hat{\theta}_+ 
    - \frac{B_0 R^2}{l r_-}\hat{\theta}_- .
\end{align}

Note that
\begin{align}
    &\frac{\hat{\theta}_{\pm}}{r_{\pm}} 
    = \hat{z}\times\frac{\mathbf{r}_{\pm}}{r_{\pm}^2},
\end{align}
where
\begin{align}
    &\mathbf{r}_{\pm} = -\left(\mathbf r \pm \frac{\mathbf l}{2}\right),
\end{align}
and therefore
\begin{align}
    &r_{\pm}^2 = r^2 + \frac{l^2}{4} \mp \mathbf r \cdot \mathbf l .
\end{align}

Under the assumption
\begin{align}
    &l \ll R \leq r,
\end{align}
we have
\begin{align}
    &\frac{1}{r_{\pm}^2} \approx \frac{1}{r^2}
    \left(1 \pm \frac{\mathbf r \cdot \mathbf l}{r^2}\right).
\end{align}

Proceeding further and neglecting terms of second order in $\frac{l}{r}$, we obtain
\begin{align}
    \frac{\hat{\theta}_+}{r_+} - \frac{\hat{\theta}_-}{r_-}
    = \frac{1}{r^2}
    \left[ 2(\mathbf r \cdot \mathbf l)\hat{\theta}
    - \hat{z}\times\mathbf l \right].
\end{align}

Hence,
\begin{align}
    \mathbf B' =
    \frac{B_0 R^2}{l r^2}
    \left[ 2(\mathbf r \cdot \mathbf l)\hat{\theta}
    - \hat{z}\times\mathbf l \right].
\end{align}

The total magnetic field is therefore
\begin{align}
    &\mathbf B = \mathbf B' + \mathbf B_0,\\
    &\mathbf B
    = B_0 \left(1-\frac{R^2}{r^2}\right)\cos\theta~\hat{r}
    - B_0 \left(1+\frac{R^2}{r^2}\right)\sin\theta~\hat{\theta}.
    \label{eq24}
\end{align}

\end{tcolorbox}


Equation~\ref{eq24} is precisely our final result, from which we may confidently assert that the solution to the main problem is
\begin{empheq}[box=\fbox]{align}
    \mathbf{v} &= v_0 \left( 1 - \frac{R^2}{r^2} \right) \cos \theta \, \hat{r} 
    - v_0 \left( 1 + \frac{R^2}{r^2} \right) \sin \theta \, \hat{\theta}.
\end{empheq}

A quick verification shows that the result obtained indeed satisfies the boundary conditions given in Eq.~\ref{eq3}.

At this point, one might question whether, in the magnetostatic problem considered earlier, choosing a superconducting half-cylinder attached to a superconducting plane would be more appropriate than a full cylinder. Indeed, in that case, the differential equations and boundary conditions of the fluid-mechanics and magnetostatic problems are formally identical throughout the entire space, rather than only in the region $y \geq 0$. Nevertheless, we may assert that the final result remains unchanged regardless of the chosen configuration. This follows from the symmetry of the system; more precisely, Eq.~\ref{eq24} shows that the magnetic field possesses only a tangential component with respect to the plane $(\Sigma): y=0$. Consequently, the superconducting plane merely induces surface currents that cancel this tangential component, without contributing any magnetic field in the region $y \geq 0$.
